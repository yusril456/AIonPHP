\section{Sejarah Machine Learning}
Sejak pertama kali komputer diciptakan manusia sudah memikirkan bagaimana caranya agar komputer dapat belajar dari pengalaman. Hal tersebut terbukti pada tahun 1952, Arthur Samuel menciptakan 
sebuah program, game of checkers, pada sebuah komputer IBM. Program tersebut dapat mempelajari gerakan untuk memenangkan permainan checkers dan menyimpan gerakan tersebut kedalam memorinya.
Istilah machine learning pada dasarnya adalah proses komputer untuk belajar dari data (learn from data). Tanpa adanya data, komputer tidak akan bisa belajar apa-apa. Oleh karena itu jika kita ingin belajar machine learning, pasti akan terus berinteraksi dengan data. Semua pengetahuan machine learning pasti akan melibatkan data. Data bisa saja sama, akan tetapi algoritma dan pendekatan nya berbeda-beda untuk mendapatkan hasil yang optimal.
\section{Belajar Machine Learning}
Machine Learning merupakan salah satu cabang dari disiplin ilmu kecerdasan buatan (Artificial Intellegence) yang membahas mengenai pembangunan sistem yang berdasarkan pada data. Banyak hal yang dipelajari, akan tetapi pada dasarnya ada 4 hal pokok yang dipelajari dalam machine learning.
\begin{enumerate}
	\item pembelajaran terarah (Supervised Learning)
	\item pembelajaran tak terarah (Unsupervised Learning)
	\item Pembelajaran semi terarah (Semi-supervised Learning)
	\item Reinforcement Learning
\end{enumerate}	]
