%%%%%%%%%%%%%%
%% Run LaTeX on this file several times to get Table of Contents,
%% cross-references, and citations.

%% If you have font problems, you may edit the w-bookps.sty file
%% to customize the font names to match those on your system.

%% w-bksamp.tex. Current Version: Feb 16, 2012
%%%%%%%%%%%%%%%%%%%%%%%%%%%%%%%%%%%%%%%%%%%%%%%%%%%%%%%%%%%%%%%%
%
%  Sample file for
%  Wiley Book Style, Design No.: SD 001B, 7x10
%  Wiley Book Style, Design No.: SD 004B, 6x9
%
%
%  Prepared by Amy Hendrickson, TeXnology Inc.
%  http://www.texnology.com
%%%%%%%%%%%%%%%%%%%%%%%%%%%%%%%%%%%%%%%%%%%%%%%%%%%%%%%%%%%%%%%%

%%%%%%%%%%%%%
% 7x10
%\documentclass{wileySev}

% 6x9
\documentclass{wileySix}

\usepackage{graphicx}
\usepackage{listings}

\usepackage{color}
 
\definecolor{codegreen}{rgb}{0,0.6,0}
\definecolor{codegray}{rgb}{0.5,0.5,0.5}
\definecolor{codepurple}{rgb}{0.58,0,0.82}
\definecolor{backcolour}{rgb}{0.95,0.95,0.92}
 
\lstdefinestyle{mystyle}{
    backgroundcolor=\color{backcolour},   
    commentstyle=\color{codegreen},
    keywordstyle=\color{magenta},
    numberstyle=\tiny\color{codegray},
    stringstyle=\color{codepurple},
    basicstyle=\footnotesize,
    breakatwhitespace=false,         
    breaklines=true,                 
    captionpos=b,                    
    keepspaces=true,                 
    numbers=left,                    
    numbersep=5pt,                  
    showspaces=false,                
    showstringspaces=false,
    showtabs=false,                  
    tabsize=2,
    language=sh
}
 
\lstset{style=mystyle}

%%%%%%%
%% for times math: However, this package disables bold math (!)
%% \mathbf{x} will still work, but you will not have bold math
%% in section heads or chapter titles. If you don't use math
%% in those environments, mathptmx might be a good choice.

% \usepackage{mathptmx}

% For PostScript text
\usepackage{w-bookps}

%%%%%%%%%%%%%%%%%%%%%%%%%%%%%%%%%%%%%%%%%%%%%%%%%%%%%%%%%%%%%%%%
%% Other packages you might want to use:

% for chapter bibliography made with BibTeX
% \usepackage{chapterbib}

% for multiple indices
% \usepackage{multind}

% for answers to problems
% \usepackage{answers}

%%%%%%%%%%%%%%%%%%%%%%%%%%%%%%
%% Change options here if you want:
%%
%% How many levels of section head would you like numbered?
%% 0= no section numbers, 1= section, 2= subsection, 3= subsubsection
%%==>>
\setcounter{secnumdepth}{3}

%% How many levels of section head would you like to appear in the
%% Table of Contents?
%% 0= chapter titles, 1= section titles, 2= subsection titles, 
%% 3= subsubsection titles.
%%==>>
\setcounter{tocdepth}{2}

%% Cropmarks? good for final page makeup
%% \docropmarks

%%%%%%%%%%%%%%%%%%%%%%%%%%%%%%
%
% DRAFT
%
% Uncomment to get double spacing between lines, current date and time
% printed at bottom of page.
% \draft
% (If you want to keep tables from becoming double spaced also uncomment
% this):
% \renewcommand{\arraystretch}{0.6}
%%%%%%%%%%%%%%%%%%%%%%%%%%%%%%

%%%%%%% Demo of section head containing sample macro:
%% To get a macro to expand correctly in a section head, with upper and
%% lower case math, put the definition and set the box 
%% before \begin{document}, so that when it appears in the 
%% table of contents it will also work:

\newcommand{\VT}[1]{\ensuremath{{V_{T#1}}}}

%% use a box to expand the macro before we put it into the section head:

\newbox\sectsavebox
\setbox\sectsavebox=\hbox{\boldmath\VT{xyz}}

%%%%%%%%%%%%%%%%% End Demo


\begin{document}


\booktitle{Machine Learning Dengan PHP }
%\booktitle{Pada Pemrograman PHP}
\subtitle{Teori dan Praktek}

\authors{Muhammad Yusril Helmi Setyawan\\
\affil{Informatics Research Center}
%Floyd J. Fowler, Jr.\\
%\affil{University of New Mexico}
}

\offprintinfo{Machine Learning Dengan PHP, Cetakan Pertama}{Muhammad Yusril Helmi Setyawan}

%% Can use \\ if title, and edition are too wide, ie,
%% \offprintinfo{Survey Methodology,\\ Second Edition}{Robert M. Groves}

%%%%%%%%%%%%%%%%%%%%%%%%%%%%%%
%% 


\titlepage


\begin{copyrightpage}{2019}
\input{info/copyrightpage}
\end{copyrightpage}

\dedication{`Jika Kamu tidak dapat menahan lelahnya belajar, 
Maka kamu harus sanggup menahan perihnya Kebodohan.'
~Imam Syafi'i~}

\begin{contributors}
\input{info/contributors}
\end{contributors}

\contentsinbrief
\tableofcontents
\listoffigures
\listoftables
\lstlistoflistings


\begin{foreword}
\input{info/foreword}
\end{foreword}

\begin{preface}
\input{info/preface}
\end{preface}


\begin{acknowledgments}
\input{info/acknowledgments}
\end{acknowledgments}

\begin{acronyms}
\input{info/acronyms}
\end{acronyms}

\begin{glossary}
\input{info/glossary}
\end{glossary}

\begin{symbols}
\input{info/symbols}
\end{symbols}

\begin{introduction}
\input{info/introduction}
\end{introduction}

%%%%%%%%%%%%%%%%%%Isi Buku_

\chapter{Pengantar Machine learning}
\section{Pengantar Machine Learning}
Machine Learning adalah ilmu komputer yang bisa bekerja tanpa diprogram secara eksplisit.Banyak peneliti berpikir bagaimana cara untuk membuat kemajuan menuju AI terhadap tingkat manusia. Machine learning ini merupakan kecerdasan buatan yang mempelajari bagaimana membuat data. Machine learning ini biasa disingkat dengan ML. Ini dibutuhkan untuk menerapkan teknik yang cepat dan kuat dalam menemukan masalah baru.
Akhirnya, pemakaian teknik ini berkaitan dengan pembelajaran mesin dan AI. Mesin ini membuktikan kepada algoritma atau program yang berjalan di komputer. Oleh karena itu, jika kita ingin belajar machine learning, pastikan anda terus berinteraksi dengan data. Semua pengetahuan machine learning pasti akan melibatkan data. Dari pada penasaran, langsung aja ikutin ulasan berikut.
\subsection{Apa itu Machine Learning?}
Machine learning adalah aplikasi dari disiplin ilmu kecerdasan buatan (Artificial Intelligence) yang menggunakan teknik statistika untuk menghasilkan suatu model otomatis dari sekumpulan data, dengan tujuan memberikan komputer kemampuan untuk “belajar”. Pembelajaran mesin atau machine learning memungkinkan komputer mempelajari sejumlah data (learn from data) sehingga dapat menghasilkan suatu model untuk melakukan proses input-output tanpa menggunakan kode program yang dibuat secara eksplisit. Proses belajar tersebut menggunakan algoritma khusus yang disebut machine learning algorithms. Terdapat banyak algoritma machine learning dengan efesiensi dan spesifikasi kasus yang berbeda-beda.
\subsection{Definisi Menurut Ahli}
Learning to recognize spoken words (Lee, 1989; Waibel, 1989).
Learning to drive an autonomous vehicle (Pomerleau, 1989).
Learning to classify new astronomical structures (Fayyad et al., 1995).
Learning to play world-class backgammon (Tesauro 1992, 1995).

\subsection{Ruang Lingkup Machine Learning}
\begin{itemize}
\item Statistik: Cara terbaik untuk menggunakan sampel yang diambil dari distribusi probabilitas diketahui untuk membantu memutuskan dari mana distribusi beberapa sampel baru diambil.
\item Brain Models: Elemen Non-linear dengan pembobotan input (Jaringan Syaraf Tiruan) telah diusulk
\item Teori Kontrol Adaptif: Bagaimana menangani pengendalian proses memiliki parameter yang tidak diketahui yang harus diestimasi selama operasi.
\item Psikologi: Bagaimana model kinerja manusia pada berbagai tugas-tugas belajar
\item Artificial Intelligence: Bagaimana menulis algoritma untuk memperoleh kecerdasan otak manusia.
\item Model evolusi: Bagaimana model aspek-aspek tertentu dari evolusi biologis untuk meningkatkan kinerja program komputer?
\end{itemize}

\subsection{Konsep Dasar Machine Learning}
Konsep tersebut meliputi kemampuan suatu individu dalam meningkatkan kecerdasan tersebut untuk belajar tanpa terkecuali pada sebuah mesin. Mesin yang mampu belajar, akan meningkatkan produktivitas manusia. Maka ia juga akan memiliki kekuatan yang mungkin tidak dimiliki mesin lainnya.
\subsection {Pentingnya ML}
\begin{itemize}
\item Beberapa tugas tidak dapat didefinisikan dengan baik, kecuali oleh contoh. (misalnya, mengenali orang). 
\item Hubungan dan korelasi dapat tersembunyi dalam suatu data yang berjumlah besar. Machine Learning / Data Mining mungkin dapat menemukan hubungan-hubungan ini.
\item Desainer Manusia sering menghasilkan mesin yang tidak bekerja dengan baik seperti yang diinginkan dalam lingkungan di mana mesin tersebut digunakan.
\item Jumlah pengetahuan yang tersedia tentang tugas-tugas tertentu mungkin terlalu besar untuk pengkodean eksplisit oleh manusia (misalnya, diagnostik medis).
\end{itemize}

\subsection{Bagian Machine Learning}
Ketika Anda melihat situs web yang kompleks seperti Facebook, Amazon, atau Netflix, kemungkinan besar situs ini berisi beberapa model Machine Learning. Dari model yang didapatkan, kita dapat melakukan prediksi yang berbeda, tergantung pada tipenya. Jika hasil prediksi bersifat diskrit, maka dinamakan proses klasifikasi. Sistem pembelajaran mesin terdiri dari tiga bagian utama, yaitu:
\begin{enumerate}
\item Model: sistem yang membentuk prediksi atau identifikasi.
\item Parameter: sinyal atau faktor yang digunakan oleh model untuk membentuk keputusannya.
\item Pemelajaran: sistem yang menyesuaikan parameter dan model dalam prediksi versus hasil aktual.
\end{enumerate}
\subsection{Cara Kerja Machine Learning}
Machine learning memiliki dua jenis teknik: Supervised Learning, yang melatih model pada data input dan output yang diketahui sehingga dapat memprediksi keluaran masa depan dan Unsupervised Learning, yang menemukan pola tersembunyi atau struktur intrinsik pada data masukan.
Penerapan metode Machine Learning dalam beberapa tahun terakhir telah berkembang di mana-mana dalam kehidupan sehari-hari. Machine Learning bukanlah hal baru dalam lanskap ilmu komputer. Machine Learning mengaitkan proses struktural dimana setiap bagian menciptakan versi mesin yang lebih baik.


\subsection{Cara Belajar ML}
Cara belajar program machine learning mengikuti cara belajar manusia, yakni belajar dari contoh-contoh. Machine learning akan mempelajari pola dari contoh-contoh yang dianalisa, untuk menentukan jawaban dari pertanyaan-pertanyaan berikutnya.
Memang tidak semua masalah bisa dipecahkan dengan program machine learning. Namun, seringkali algoritma yang sifatnya kompleks, ternyata bisa dipecahkan dengan sangat simpel oleh machine learning. Beberapa contoh program machine learning yang telah digunakan dalam kehidupan sehari-hari:
\begin{itemize}
\item Pendeteksi Spam
\item Pendeteksi Wajah
\item Rekomendasi Produk
\item Asisten Virtual
\item Diagnosa Medis
\item Pendeteksi Penipuan Kartu Kredit
\item Pengenal Digit
\item Perdagangan Saham
\item Segmentasi Pelanggan
\item Mobil yang bisa Mengendarai Sendiri
\end{itemize}
\begin{itemize}
\item Supervised Learning
Pembelajaran mesin yang diawasi menciptakan model yang melancarkan prediksi berdasarkan bukti adanya ketidakpastian. Algoritma pembelajaran yang diawasi memerlukan seperangkat data masukan dan tanggapan yang diketahui terhadap data (output) dan melatih model untuk menghasilkan prediksi yang masuk akal untuk respon terhadap data baru. Gunakan pembelajaran ini jika Anda ingin mengetahui data output yang ingin Anda prediksi. Pembelajaran ini diawasi menggunakan teknik klasifikasi dan regresi untuk mengembangkan model prediktif.
Teknik klasifikasi memprediksi respons diskrit – misalnya, apakah email itu asli atau spam, atau apakah tumor itu kanker atau tidak. Model klasifikasi mengklasifikasikan data masukan ke dalam kategori tersebut. Aplikasi yang umum termasuk pencitraan medis. Misalnya aplikasi untuk pengenalan tulisan, maka anda harus menggunakan klasifikasi untuk mengenali huruf dan angka.
Jika Anda bisa melakukannya, Anda memiliki landasan yang dapat Anda gunakan pada satu dataset ke dataset yang akan dicoba lagi selanjutnya. Anda bisa mengisi waktu seperti mempersiapkan data lebih lanjut dan memperbaiki hasilnya nanti, begitu Anda lebih percaya diri. Dalam pengolahan citra dan penglihatan komputer, teknik pengenalan pola tanpa pemeriksaan digunakan untuk deteksi objek dan segmentasi. Algoritma yang umum mengadakan klasifikasi yang meliputi dukungan mesin vektor (SVM).
Supervised Learning terbagi atas 2 bagian :
\begin{enumerate}
\item Regression
Data yang ada diberikan real value, numerical atau floating point, agar dapat mencoba mendeteksi harga saham di kemudian hari. Contoh: time series data dari harga saham berdasarkan waktu.
\item Classification (Discrete/ Category)
Data yang ada diberikan label atau kategori, agar dapat diambil keputusan berdasarkan label/ kategori tersebut.
\end{enumerate}


\item Unsupervised Learning
Ini menemukan pola tersembunyi atau struktur intrinsik dalam data. Ini digunakan untuk menarik kesimpulan dari kumpulan data yang terdiri dari data masukan tanpa respon berlabel. Clustering adalah teknik belajar tanpa pengamatan yang umum. Ini digunakan untuk analisis data eksplorasi dalam menemukan pola atau pengelompokan tertutup dalam data. Aplikasi untuk analisis cluster meliputi analisis urutan gen, riset pasar dan pengenalan objek.
Misalnya, jika sebuah perusahaan telepon seluler ingin mengoptimalkan lokasi di mana mereka membangun menara telepon seluler, mereka dapat menggunakan pembelajaran mesin untuk memperkirakan jumlah kelompok orang yang bergantung pada menara mereka. Telepon hanya bisa berbicara dengan satu menara sekaligus, sehingga tim menggunakan algoritma pengelompokan untuk merancang peletakan menara seluler terbaik dalam mengoptimalkan penerimaan sinyal bagi kelompok dan dari pelanggan mereka.
Algoritma yang umum mengadakan clustering meliputi k-means dan k-medoids, hirarki clustering, model campuran Gaussian, model Markov tersembunyi, peta pengorganisasian sendiri, clustering fuzzy c-means dan clustering subtraktif.
\end{itemize}
\subsection {Metode Algoritma Machine Learning} 
\begin{enumerate}
\item Supervised machine learning algorithms
Supervised machine learning adalah algoritma machine learning yang dapat menerapkan informasi yang telah ada pada data dengan memberikan label tertentu, misalnya data yang telah diklasifikasikan sebelumnya (terarah). Algoritma ini mampu memberikan target terhadap output yang dilakukan dengan membandingkan pengalaman belajar di masa lalu.
\item Unsupervised machine learning algorithms
Unsupervised machine learning adalah algoritma machine learning yang digunakan pada data yang tidak mempunyai informasi yang dapat diterapkan secara langsung (tidak terarah). Algoritma ini diharapkan mampu menemukan struktur tersembunyi pada data yang tidak berlabel.
\item Semi-supervised machine learning algorithms
Semi-supervised machine learning adalah algoritma yang digunakan untuk melakukan pemebelajaran data berlabel dan tanpa label. Sistem yang menggunakan metode ini dapat meningkatkan efesiensi output yang dihasilkan.
\item Reinforcement machine learning algorithms
Reinforcement machine learning adalah algoritma yang mempunyai kemampuan untuk bertinteraksi dengan proses belajar yang dilakukan, algoritma ini akan memberikan poin (reward) saat model yang diberikan semakin baik atau mengurangi poin (error) saat model yang dihasilkan semakin buruk. Salah satu penerapannya adalah pada mesin pencari.
\end{enumerate}
\subsection{Aplikasi Machine Learning} 
Data bisa saja sama, namun untuk pendekatan terhadap algoritmanya berbeda-beda dalam hal  mendapatkan hasil yang optimal. Berikut merupakan contoh aplikasi pembelajaran mesin:
\begin{enumerate}
\item Penelusuran web: Laman peringkat berdasarkan apa yang anda klik
\item Biologi komputasional: Obat desain rasional di komputer berdasarkan eksperimen masa lalu.
\item Keuangan: tetapkan siapa yang akan mengirim kartu kredit yang ditawarkan. Evaluasi risiko pada penawaran kredit dan bagaimana cara memutuskan dimana menginvestasikan uangnya.
\item E-commerce: Memprediksi customer churn. Apakah transaksi itu salah atau tidak.
\item Eksplorasi ruang angkasa: Menyelidiki ruang angkasa dan astronomi radio.
\item Robotika: Bagaimana menangani ketidakpastian di lingkungan baru. Seperti otonom dan Mobil self-driving.
\item Pengambilan informasi: Ajukan pertanyaan melalui database di seluruh web.
\item Jaringan sosial: Data tentang hubungan dan preferensi. Mesin belajar mengekstrak nilai dari data.
\item Debugging: Ini didunakan dalam masalah ilmu komputer seperti debugging.
\end{enumerate}
Dari model yang didapatkan, kita dapat melakukan prediksi yang dibedakan menjadi dua macam, tergantung tipe keluarannya. Jika hasil prediksi bersifat diskrit, maka ini dinamakan proses klasifikasi. Salah satu teknik pengaplikasian machine learning adalah supervised learning. Seperti yang dibahas sebelumnya, machine learning tanpa data ini tidak akan bisa bekerja.
\subsection{Dampak Machine Learning di Masyarakat}  
Dalam penerapan teknologi machine learning ini, kebanyakan orang mungkin telah merasakan dampaknya sekarang. Dalam pengembangan teknologi machine learning ada dampak yang saling bertolak belakang yaitu dampak negatif dan dampak positif. Ini yang akan memberikan masukan yang berdampak buruk dan baiknya, tergantung terhadap orang yang menilainya. Akan tetapi semua ini tidak selalu berjalan dengan mulus.
Dampak positif dari machine learning adalah mendapat kesempatan bagi para wirausahawan dan praktisi teknologi untuk terus berkreasi dalam mengembangkan machine learning. Tentunya untuk membantu aktivitas manusia sebagi sesuatu yang menguntungkan. Itulah salah satu dampak positif dari machine learning. Contohnya adalah untuk pengecekan ejaan untuk tiap bahasa yang ada dalam microsoft Word.
Pengecekan manual akan menghabiskan waktu untuk beberapa hari, juga memerlukan banyak tenaga untuk mendapatkan penulis yang sempurna. Namun, dengan bantuan fitur pengecekan tersebut, maka secara real-time kesalahan yang terjadi saat pengetikan kita bisa langsung melihatnya.
Dampak negatifnya kita harus waspada. Yang takut di khawatirkan yaitu adanya pengurangan tenaga kerja. Kenapa? Karena pekerjaan yang seharusnya di kerjakan oleh banyak orang, sekarang telah digantikan oleh alat teknologi yang disebut sebagai machine learning. Hal tersebut merupakan suatu permasalahan yang akan kita hadapi. Ditambah dengan ketergantungan terhadap teknologi yang semakin banyak dan berkembang di kehidupan kita. Kadang manusia lebih nyaman dengan perkembangan teknologi sekarang ini seperti gadget.




\chapter{Teknologi dengan Machine Learning}
\section{Sejarah Machine Learning}
Sejak pertama kali komputer diciptakan manusia sudah memikirkan bagaimana caranya agar komputer dapat belajar dari pengalaman. Hal tersebut terbukti pada tahun 1952, Arthur Samuel menciptakan 
sebuah program, game of checkers, pada sebuah komputer IBM. Program tersebut dapat mempelajari gerakan untuk memenangkan permainan checkers dan menyimpan gerakan tersebut kedalam memorinya.
Istilah machine learning pada dasarnya adalah proses komputer untuk belajar dari data (learn from data). Tanpa adanya data, komputer tidak akan bisa belajar apa-apa. Oleh karena itu jika kita ingin belajar machine learning, pasti akan terus berinteraksi dengan data. Semua pengetahuan machine learning pasti akan melibatkan data. Data bisa saja sama, akan tetapi algoritma dan pendekatan nya berbeda-beda untuk mendapatkan hasil yang optimal.
\begin{enumerate}
	\item pembelajaran terarah (Supervised Learning)
	\item pembelajaran tak terarah (Unsupervised Learning)
	\item Pembelajaran semi terarah (Semi-supervised Learning)
	\item Reinforcement Learning
\end{enumerate}
\section{Dampak Machine Learning di Masyarakat}
Penerapan teknologi machine learning mau tidak mau pasti telah dirasakan sekarang. Setidaknya ada dua dampak yang saling bertolak belakang dari pengembangan teknolgi machine learning. Ya, dampak positif dan dampak negatif.Salah satu dampak positif dari machine learning adalah menjadi peluang bagi para wirausahawan dan praktisi teknologi untuk terus-menerus berkarya dalam mengembangkan sebuah bidang teknologi machine learning. Terbantunya aktivitas yang harus dilakukan manusia pun menjadi salah satu dampak positif machine learning. Sebagai contohnya adalah adanya fitur pengecekan ejaan untuk tiap bahasa pada Microsoft Word. Pengecekan secara manual akan memakan waktu berhari-hari dan melibatkan banyak tenaga untuk mendapatkan penulisan yang sempurna. Tapi dengan bantuan fitur pengecekan ejaan tersebut, secara real-time kita bisa melihat kesalahan yang terjadi pada saat pengetikan.
Akan tetapi disamping itu ada dampak negatif yang harus kita waspadai. Adanya pemotongan tenaga kerja karena pekerjaan telah digantikan oleh alat teknologi machine learning adalah suatu permasalahan yang harus dihadapi. Ditambah dengan ketergantungan terhadap teknologi akan semakin terasa. Manusia akan lebih terlena oleh kemampuan gadget-nya sehingga lupa belajar untuk melakukan suatu aktivitas tanpa bantuan teknologi.
\section{Deep Learning}
Dalam istilah praktis, deep learning merupakan bagian dari machine learning. Sebuah model machine learning perlu 'diberitahu' untuk bagaimana ia menciptakan prediksi akurat, dengan terus diberikan data. Sementara model deep learning dapat mempelajari metode komputasinya sendiri, dengan 'otaknya' sendiri, apabila diibaratkan.Sebuah model deep learning dirancang untuk terus menganalisis data dengan struktur logika yang mirip dengan bagaimana manusia mengambil keputusan. Untuk dapat mencapai kemampuan itu, deep learning menggunakan struktur algoritma berlapis yang disebut artificial neural network (ANN). Dikutip dari Zendeks, desain ANN terinspirasi dari jaringan neural biologis dari otak manusia. Hal ini membuat mesin kecerdasannya menjadi jauh lebih tangguh dibandingkan model machine learning standar. Rumit memang untuk memastikan model deep learning yang diciptakan tidak memberikan kesimpulan yang tidak tepat. Tapi ketika ia telah bekerja dengan benar, maka fungsi deep learning akan menjadi terobosan yang berpotensi menjadi tulang belakang sebuah kecerdasan buatan sebenarnya. Data-data yang digunakan dalam sebuah deep learning sangatlah penting, karena semakin banyak datanya, maka semakin banyak yang bisa dipahami model deep learning tersebut. Contoh dari penggunaan model deep learning bisa dilihat dari AlphaGo-nya. Google menciptakan program komputer yang belajar bermain sebuah game sejenis catur dari China bernama Go. Tentunya, game ini membutuhkan pemikiran dan intuisi yang tajam untuk menang. Dengan bermain melawan pemain Go profesional, deep learning AlphaGo mempelajari bagaimana ia bermain di tingkat yang belum terjamah sebelumnya dalam kecerdasan buatan. Hebatnya, apa yang dilakukannya tanpa instruksi apapun ketika melancarkan gerakan-gerakan spesifik. Saat si pemain AlphaGo berhasil mengalahkan sejumlah pemain Go 'nyata' dunia, dunia melihat bagaimana cerdasnya sebuah mesin yang bahkan bisa mengungguli manusia.
\section{Teknologi}
Teknologi adalah berbagai keperluan serta sarana berbentuk aneka macam peralatan atau sistem yang berfungsi untuk memberikan kenyamanan serta kemudahan bagi manusia.Teknologi berasal dari kata technologia (bahasa Yunani) techno artinya ‘keahlian’ danlogia artinya ‘pengetahuan’. Pada awalnya makna teknologi terbatas pada benda- benda berwujud seperti peralatan- peralatan atau mesin. Perkembangan teknologi adalah perubahan sistematis yang terjadi terhadap teknologi. Selama beri-ribu tahun lalu teknologi sudah dikenal oleh manusia, hanya saja bentuk- bentuknya tidak secanggih dengan apa yang kita temukan di masa kini.
\begin{enumerate}
	\item Masa pra-Sejarah
Pada masa pra sejarah ini, teknologi yang digunakan terbuat dari batu, perunggu dan besi. Teknologi yang dikenal di zaman pra-sejarah contohnya adalah Pedang, kapak genggam dan bejana perunggu.
	\item Teknologi Jaman Kuno
Pada masa ini teknologi sudah berkembang ke arah kontruksi, maritim, pertanian dan alat- alat tulis. Manusia sudah mengenal bagaimana membangun sebuah kontruksi bangunan sampai pada tahap rumit. Contohnya Piramid, Kapal, Mercusuar dan jam matahari.
	\item Teknologi Abad Pertengahan Hingga era Modern
Pada masa ini teknologi yang digunakan sudah mulai mengalami kemajuan, hal ini ditandai dengan adanya berbagai penemuan, seperti di bidang astronomi, medis, matematika, militer hingga ilmu kartografi. Contohnya busur silang, mesin cetak, aljabar dan navigasi kapal.
	\item Teknologi era Revolusi Industri
Perkembangan teknologi mulai terlihat semakin jelas di masa ini. Berbagai jenis mesin berhasil dibuat yang kemudian menggantikan tenaga manusia menjadi tenaga mesin. Masa ini adalah cikal bakal perkembangan teknologi di masa kini. Contohnya Mobil generasi awal, telegrap, telepon, mesin tenun, mesin uap dan sepeda.
	\item Teknologi di Abad 20
Pada masa ini. Neil Amstrong berhasil mendarat di bulan. Teknologi dalam bidang lain pun berkembang pesat. Dalam bidang militer, bom atom berhasil diciptakan. Transistor yang menjadi cikal bakal ukuran komputer kecil seperti sekarang ini juga ditemukan. Pada akhir abad ini Internet mulai diperkenalkan untuk umum dan komersil. Contoh teknologi lain Abad 20: Kulkas, Teknologi vaksinasi, vakum, microwave.
	\item Perkembangan Teknologi Abad 21
Pada masa ini, berbagai teknologi sudah mulai dikembangkan. Mulai dari teknologi yang dibutuhkan untuk rumah tangga, pendidikan, sosial, teknologi informasi, dan hal lainnya perkembangan dapat dilihat dari aneka inovasi teknologi yang ada saat ini. Kemajuan teknologi menyentuh berbagai macam sektor, mulai dari :
\end{enumerate}
\section{Teknologi Dalam Bidang Ekonomi}
Kemajuan teknologi di bidang ekonomi ini berupa perkembangan sistem keuangan yang digunakan. Jika dahulu orang melakukan bertransaksi secara real atau nyata, atau berhadapan antara pembeli dengan penjual, maka kini beralih menjadi online. Selain itu, sistem keuangan juga jadi berubah menjadi e-money.
\section{Teknologi Pangan}
Sistem pertanian yang ada saat ini tentunya berbeda dengan sistem pertanian pada zaman dahulu, mulai dari bibir, sistem tanam, serta teknik menanamnya.
\section{Teknologi Informasi}
Kemajuan informasi ini ditandai dengan mudahnya masyarakat dalam memperoleh atau mendapatkan informasi melalui internet dengan berbagai perangkat teknologi yang ada.
\section{Teknologi Komunikasi}
Kemajuan komunikasi ini ditandai dengan mudahnya seseorang untuk berkomunikasi dengan orang lain, walau dengan jarak yang cukup jauh.
\section{Teknologi Transportasi}
Salah satu kemajuan dalam bidang transportasi ini adalah adanya berbagai macam alat transportasi modern, yang mempermudah seseorang untuk mengangkut barang atau bepergian dari 1 temat ke tempat lain dengan mudah
\section{Teknologi Medis}
Salah satu kemajuan dalam dunia medis ini adalah ditemukannya berbagai macam vaksin guna mencegah berbagai macam penyakit berbahaya.
\section{Teknologi Pendidikan}
Adapun teknologi yang turut berkembang dalam dunia pendidikan adalah, berkembangnya sistem pendidikan jadi lebih baik, tenaga pendidik serta murid mudah memahami berbagai pelajaran yang diberikan, dll.







\chapter{PHP sebagai Bahasa Pemrograman}
\section{PHP sebagai Bahasa Pemprograman}
\subsection{Mengenal PHP sebagai Bahasa Pemprograman}
PHP adalah singkatan dari "PHP: Hypertext Prepocessor", yaitu bahasa pemrograman yang digunakan secara luas untuk penanganan pembuatan dan pengembangan sebuah situs web dan bisa digunakan bersamaan dengan HTML. PHP diciptakan oleh Rasmus Lerdorf pertama kali tahun 1994. Sejak versi 3.0, nama bahasa ini diubah menjadi "PHP: Hypertext Prepocessor" dengan singkatannya "PHP". PHP versi terbaru adalah versi ke-5.
Bahasa pemrograman ini termasuk ke dalam bahasa pemrograman yang serba guna dan mendukung terhadap PHP Code, Text, HTML, CSS, dan JavaScript. Bahasa pemrograman PHP juga mampu menangani banyak hal dalam pengembangan web.
Bahasa ini mampu mengumpulkan data serta membuat konten laman web menjadi lebih dinamis. Bahasa ini dapat digunakan untuk membuat, membuka, membaca, menulis, dan menutup file yang berada di sisi server. Bahasa pemrograman ini juga dapat menangani database, seperti menghapus, menambah, atau memodifikasi data.
Tidak hanya itu, bahasa ini juga dapat menangani keamanan dari data. Bahasa pemrograman ini dapat digunakan untuk membatasi pengguna untuk mengakses beberapa laman pada website yang dikembangkan. Bahasa pemrograman ini juga mampu mengenkripsi data yang ada.
\subsection{Sejarah PHP}
Pada awalnya PHP dikenal dengan singkatan Personal Home Page. Karena server tersebut di peruntukan untuk website pribadi. Tetapi untuk saat ini PHP sudah bermetamorfosis menjadi bahasa pemrograman yang sangat populer yang digunakan untuk website terkenal seperti Wikipedia,wordpress,joomla,dll.
Untuk saat ini php dikenal dengan singkatan Hypertext Preprocessor sebuah kepanjangan rekursif, yakni permainan kata dimana kepanjangannya terdiri dari singkatan itu sendiri. Bahasa pemrograman php banyak digunakan karena sifatnya yang open source yaitu dapat digunakan secara gratis.Bahasa Pemrograman PHP adalah bahasa pemrograman script server-side yang didesain untuk pengembangan web. Selain itu, PHP juga bisa digunakan sebagai bahasa pemrograman umum. Pertama kali di kembangkan oleh Rasmust Lerdorf pada tahun 1995, dan sekarang php dikembangkan oleh The PHP Group.
Sementara untuk penyisipan kode php dapat disisipkan pada html. Karena php bersifat Scripting Language atau Bahasa Pemprograman script. PHP sendiri memiliki perkembangan versi dari tahun ketahun di antaranya :
  \begin{enumerate}
     \item PHP/ FI : Personal Home Page / Forms Interfreter.
      Berasal dari tahun 1994 yang dikembangkan oleh Rasmus Lerdoft untuk membuat kode program (script) dengan Bahasa perl untuk web pribadinya. Salah satu kegunaan script ini adalah untuk menampilkan resume pribadi dan mencatat jumlah pengunjung ke sejumlah website.
     \item PHP/ FI : Personal Home Page / Form Interpreter 2
      Pada 1996 Rasmus Lerdoft mengumumkan PHP/FI versi 2.0. versi 2 ini dirancang lerdoft pada saat mengerjakan sebuah proyek di University of Toronto yang membutuhkan pengolahan data dan tampilan web yang rumit.
     \item PHP : Hypertext Prepocessor 3
      Terjadi pada pertengahan tahun 1997, telah banyak menarik perhatian programmer namun Bahasa ini memiliki masalah dengan kestabilan yang kurang bisa diandalkan.
     \item PHP : Hypertext Preprocessor 4
      Dalam fitur ini PHP memperkenalkan beberapa fitur lanjutan, seperti layer abstraksi antara PHP dan web server, menambahkan mekanisme thread-safety, dan two-stage parsing.
    \item PHP : Hypertext Preprocessor 5
      Versi PHP terakhir hingga saat ini, yaitu PHP 5.X diluncurkan pada 13 juli 2004. PHP 5 telah mendukung penuh pemrograman object dan peningkatan perfoma melalui Zend engine versi 2.
    \item PHP Hypertext Preprocessor 7
      Pada versi ini programmer masih kebingungan karena terjadi peloncatan versi dari versi 5 ke versi 7. PHP berkembang dari proyek experimen yang dinamakan PHPNG(PHP Next Generation). Proyek PHPNG bertujuan untuk menulis ulang kode PHP untuk meningkatkan perfoma. Dari proyek ini perfoma ini berhasil 100\% dari versi sebelumnya sehingga menamainya versi 7.
 \end{enumerate}
\subsection{Kelebihan Bahasa Pemrograman PHP}
  \begin{itemize}
    \item Mudah dan Serba Guna
      Seperti yang telah dijelaskan sebelumnya, kelebihan bahasa pemrograman PHP ini salah satunya adalah mudah untuk digunakan dan mendukung banyak kegunaan. Bahasa pemrograman ini dapat digunakan dengan mudah untuk membuat sisi server dari laman yang kita kembangkan dan penggunaan lainnya. Bahkan, bahasa pemrograman ini mendukung berbagai macam bahasa lain, seperti CSS dan JavaScript.
      Dan untuk mengembangkan produk dengan menggunakan bahasa pemrograman ini, telah terdapat banyak framework yang dapat digunakan. Salah satunya adalah Laravel yang menjadi framework PHP populer di kalangan developer web. Kelebihan ini tentu akan memberikan kemudahan bagi para developer dalam membangun produk-produk berbasis bahasa pemrograman PHP.
    \item Memiliki Komunitas yang Besar
      Salah satu kriteria bahasa pemrograman yang tepat untuk dipelajari adalah bahasa pemrograman tersebut harus memiliki komunitas yang besar. Salah satu alasannya adalah karena dengan komunitas yang besar, kita dapat belajar dengan mudah dengan adanya konunitas tersebut.
      Salah satu kelebihan bahasa pemrograman PHP menawarkan komunitas yang sangat besar. Tidak hanya besar, komunitas dari bahasa pemrograman PHP merupakan komunitas yang sangat aktif. Bahkan setiap kebanyakan masalah yang ada pada bahasa pemrograman ini telah memiliki solusi yang telah ada sebelumnya.
    \item Mendukung Database
     Bahasa pemrograman PHP memiliki kelebihan yang dapat digunakan untuk menangani database dengan sangat baik. Seperti yang telah dijelaskan sebelumnya, bahasa ini dapat digunakan untuk mengedit, menambahkan, atau bahkan digunakan untuk menghapus data pada database yang kita miliki.
     Bahasa pemrograman ini dapat bekerja dengan sangat baik untuk menangani berbagai hal yang berkaitan dengan file system, output HTML, images, pdfs, swf files, dan xhtml.
  \end{itemize}
\subsection{Penerapan PHP sebagai Bahasa Pemprograman}
Fungsi bahasa pemrograman php sendiri untuk web digunakan untuk dapat menyesuaikan tanpilan konten sesuai dengan situasi. Web yang bersifat dinamis juga digunakan untuk menyimpan data ke database dengan memproses from dan juga dapat megubah tampilan website sesuai inputan dari seorang user.PHP juga banyak diaplikasikan untuk pembuatan program-program seperti sistem informasi  klinik, rumah sakit, akademik, keuangan, manajemen aset, manajemen bengkel dan lain-lain. Dapat dikatakan bahwa program aplikasi yang dulunya hanya dapat dikerjakan untuk desktop aplikasi, PHP sudah dapat mengerjakannya.
Penerapan PHP saat ini juga banyak ditemukan pada proyek-proyek pemerintah seperti e-budgetting, e-procurement, e-goverment dan e e lainnya. Website Ubaya ini juga dibuat menggunakan PHP.
PHP juga dapat dilihat sebagai pilihan lain dari ASP.NET/C\#/VB.NET Microsoft, ColdFusion Macromedia, JSP/Java Sun Microsystems, dan CGI/Perl. Contoh aplikasi lain yang lebih kompleks berupa CMS yang dibangun menggunakan PHP adalah Wordpress, Mambo, Joomla, Postnuke, Xaraya, dan lain-lain.

\subsection{Sisi lain dari PHP  sebahgai Bahasa Pemprograman}
Menurut penulis yang sejak lama terlibat dalam pembuatan program dengan PHP ini adalah :
  \begin{itemize}
    \item Bahasa pemrograman PHP adalah sebuah bahasa script yang tidak perlu untuk dikompilasi (compile)
    \item Mudah diinstall ke dalam web server yang mendukung PHP seperti apache dengan konfigurasi yang mudah
    \item Dalam sisi pengembangan lebih mudah karena banyaknya milis-milis ataupun tutorial yang membahas tentang PHP
    \item PHP dapat dijalankan diberbagai sistem operasi, baik Windows, Linux, Macintosh.
  \end{itemize}
Meskipun bahasa pemrograman PHP tidak sepopuler bahasa pemrograman lainnya, bahasa pemrograman ini merupakan sebuah bahasa yang digunakan oleh perusahaan-perusahaan ternama, seperti Facebook dan IBM.
Kelebihan bahasa pemrograman PHP ini bahkan memikat perusahaan Facebook untuk membangun platform yang dimilikinya. Tidak hanya Facebook, perusahaan penyedia layanan web, yaitu WordPress, juga menggunakan bahasa pemrograman ini untuk mengembangkan platformnya.

Tidak hanya Facebook, perusahaan penyedia layanan web, yaitu WordPress, juga menggunakan bahasa pemrograman ini untuk mengembangkan platformnya.






\chapter{Implementasi PHP pada Machine Learning}

\section{PHP-ML - Machine Learning library for PHP}
Pendekatan segar di php. machine learning Algoritma, , validasi silang , jaringan syaraf , pemroses fitur ekstraksi dan lebih banyak lagi Sebagai salah satu perpustakaan.Php-ml membutuhkan php \& gt; = 7.1.fresh pendekatan untuk machine learning di php. Algoritma, validasi silang , jaringan syaraf , pemroses fitur ekstraksi dan lebih banyak lagi sebagai salah satu perpustakaan.
Php-ml membutuhkan php \& = 7.1.
Simple example of classification dapat kita lihat pada listing \ref{lst:code1}
\lstinputlisting[caption=contoh perintah 1,label={lst:code1}]{src/code1.tex}
\section{installation} 
Saat ini perpustakaan ini dalam proses pengembangan, tapi anda bisa memasangnya dengan komposer:

\begin{verbatim}composer require php-ai/php-ml\end{verbatim}

\section{examples}
Contoh script tersedia di repository yang terpisah php-ai/php-ml-examples.


\section{Features}
Ada beberapa fitur diantaranya :
\begin{enumerate}
\item Association rule Lerning
\item Classification
\item Regression
\item Clustering
\item Metric
\item Workflow
\item Neural Network
\item Cross Validation
\item Feature Selection
\item Preprocessing
\item Feature Extraction
\item Datasets
\item Models management
\item Math
\end{enumerate}

\section{Contribute}
\begin{enumerate}
\item Guide: CONTRIBUTING.md
\item Issue Tracker: github.com/php-ai/php-ml
\item Source Code: github.com/php-ai/php-ml
\end{enumerate}

\section{Machine Learning}

\subsection{Perbedaan Data Mining/Penggalian Data}
 Data mining adalah sebuah proses untuk menemukan pengetahuan, ketertarikan, dan pola baru dalam bentuk model yang deskriptif, dapat dimengerti, dan prediktif dari data dalam skala besar. Dengan kata lain data mining merupakan ekstraksi atau penggalian pengetahuan yang diinginkan dari data dalam jumlah yang sangat besar.
Dari definisi diatas dapat disimpulkan bahwa pada pembelajaran mesin berkaitan dengan studi, desain dan pengembangan dari suatu algoritma yang dapat memampukan sebuah komputer dapat belajar tanpa harus diprogram secara eksplisit. Sedangkan pada data mining dilakukan proses yang dimulai dari data yang tidak terstruktur lalu diekstrak agar mendapatkan suatu pengetahuan ataupun sebuah pola yang belum diketahui. Selama proses data mining itulah algoritma dari pembelajaran mesin digunakan.


\subsection Tipe Machine Learning Algoritma
 Machine Learning merupakan salah satu cabang dari disiplin ilmu Kecerdasan Buatan (Artificial Intellegence) yang membahas mengenai pembangunan sistem yang berdasarkan pada data. Banyak hal yang dipelajari, akan tetapi pada dasarnya ada 4 hal pokok yang dipelajari dalam machine learning.
\begin{enumerate}
 \item Pembelajaran Terarah (Supervised Learning)membuat fungsi yang memetakan masukan ke keluaran yang dikehendaki. Misalnya pengelompokan (klasifikasi). Merupakan algoritma yang belajar berdasarkan sekumpulan contoh pasangan masukan-keluaran yang diinginkan dalam jumlah yang cukup besar. Algoritma ini mengamati contoh-contoh tersebut dan kemudian menghasilkan sebuah model yang mampu memetakan masukan yang baru menjadi keluaran yang tepat.
       \par Salah satu contoh yang paling sederhana adalah terdapat sekumpulan contoh masukan berupa umur seseorang dan contoh keluaran yang berupa tinggi badan orang tersebut. Algoritma pembelajaran melalui contoh mengamati contoh-contoh tersebut dan kemudian mempelajari sebuah fungsi yang pada akhirnya dapat “memperkirakan” tinggi badan seseorang berdasarkan masukan umur orang tersebut.
 \item Pembelajaran Tak Terarah (Unsupervised Learning) memodelkan himpunan masukan, seperti penggolongan (clustering).Algoritma ini mempunyai tujuan untuk mempelajari dan mencari pola-pola menarik pada masukan yang diberikan. Meskipun tidak disediakan keluaran yang tepat secara eksplisit. Salah satu algoritma unsupervised learning yang paling umum digunakan adalah clustering/pengelompokan .
       \par Contoh unsupervised learning dalam dunia nyata misalnya  seorang supir taksi yang secara perlahan-lahan menciptakan konsep “macet” dan “tidak macet” tanpa pernah diberikan contoh oleh siapapun .
 \item Pembelajaran Semi Terarah (Semi-supervised Learning)ipe ini menggabungkan antara Supervised dan Unsupervised untuk menghasilkan suatu fungsi.
       \par Algoritma pembelajaran semi terarah menggabungkan kedua tipe algoritma di atas, di mana diberikan contoh masukan-keluaran yang tepat dalam jumlah sedikit dan sekumpulan masukan yang keluarannya belum diketahui. Algoritma ini harus membuat sebuah rangkaian kesatuan antara dua tipe algoritma di atas untuk dapat menutupi kelemahan pada masing-masing algoritma.
       \par Misalnya sebuah sistem yang dapat menebak umur seseorang berdasarkan foto orang tersebut.  Sistem tersebut membutuhkan beberapa contoh, misalnya yang didapatkan dengan mengambil foto seseorang dan menanyakan umurnya (pembelajaran terarah). Akan tetapi, pada kenyataannya beberapa orang sering kali berbohong tentang umur mereka sehingga menimbulkan noise pada data. Oleh karena itu, digunakan juga pembelajaran tak terarah agar dapat saling menutupi kelemahan masing-masing, yaitu noise pada data dan ketiadaan contoh masukan-keluaran.
 \item Reinforcement Learning Tipe ini mengajarkan bagaimana cara bertindak untuk menghadapi suatu masalah, yang suatu tindakan itu mempunyai dampak. 
       \par Adalah sebuah algoritma pembelajaran yang diterapkan pada agen cerdas agar ia dapat menyesuaikan dengan kondisi dilingkungannya, hal ini dicapai dengan cara memaksimalkan nilai dari hadiah ‘reward’ yang dapat dicapai. Suatu hadiah didefinisikan sebuah tanggapan balik ‘feedback’ dari tindakan agen bahwa sesuatu baik terjadi.Sebagai contoh, sangatlah sulit untuk memrogram sebuah agen untuk menerbangkan sebuah helikopter, tetapi dengan memberikan beberapa nilai negatif untuk menabrak, bergoyang-goyang, serta melenceng dari jalur tujuan perlahan-lahan agen tersebut dapat belajar menerbangkan helikopter dengan lebih baik.
\end{enumerate}

\subsection{Contoh Penerapan Machine Learning}
 Contoh penerapan machine learning dalam kehidupan adalah sebagai berikut :
\begin{enumerate}
 \item Penerapan di bidang kedoteran contohnya adalah mendeteksi penyakit seseorang dari gejala yang ada. Contoh lainnya adalah mendeteksi penyakit jantung dari rekaman elektrokardiogram.
 \item Pada bidang computer vision contohnya adalah penerapan pengenalan wajah dan pelabelan wajah seperti pada facebook. Contoh lainnya adalah penterjemahan tulisan tangan menjadi teks.
 \item Pada biang information retrival contohnya adalah penterjemahan bahasa dengan menggunakan komputer, mengubah suara menjadi teks, dan filter email spam.
\end{enumerate}

Salah satu teknik pengaplikasian machine learning adalah supervised learning. Seperti yang dibahas sebelumnya, machine learning tanpa data maka tidak akan bisa bekerja. Oleh karena itu hal yang pertama kali disiapkan adalah data. Data biasanya akan dibagi menjadi 2 kelompok, yaitu data training dan data testing. Data training nantinya akan digunakan untuk melatih algoritma untuk mencari model yang cocok, sementara data testing akan dipakai untuk mengetes dan mengetahui performa model yang didapatkan pada tahapan testing.
Dari model yang didapatkan, kita dapat melakukan prediksi yang dibedakan menjadi dua macam, tergantung tipe keluarannya. Jika hasil prediksi bersifat diskrit, maka dinamakan proses klasifikasi. Contohnya klasifikasi jenis kelamin dilihat dari tulisan tangan (output laki dan perempuan). Sementara jika kelurannya bersifat kontinyu, maka dinamakan proses regresi. Contohnya prediksi kisaran harga rumah di kota Bandung (output berupa harga rumah).

\subsection{Dampak Machine Learning di Masyarakat}
Penerapan teknologi machine learning mau tidak mau pasti telah dirasakan sekarang. Setidaknya ada dua dampak yang saling bertolak belakang dari pengembangan teknolgi machine learning. Ya, dampak positif dan dampak negatif.
Salah satu dampak positif dari machine learning adalah menjadi peluang bagi para wirausahawan dan praktisi teknologi untuk terus berkarya dalam mengembangkan teknologi machine learning. Terbantunya aktivitas yang harus dilakukan manusia pun menjadi salah satu dampak positif machine learning. Sebagai contohnya adalah adanya fitur pengecekan ejaan untuk tiap bahasa pada Microsoft Word. Pengecekan secara manual akan memakan waktu berhari-hari dan melibatkan banyak tenaga untuk mendapatkan penulisan yang sempurna. Tapi dengan bantuan fitur pengecekan ejaan tersebut, secara real-time kita bisa melihat kesalahan yang terjadi pada saat pengetikan.
Akan tetapi disamping itu ada dampak negatif yang harus kita waspadai. Adanya pemotongan tenaga kerja karena pekerjaan telah digantikan oleh alat teknologi machine learning adalah suatu permasalahan yang harus dihadapi. Ditambah dengan ketergantungan terhadap teknologi akan semakin terasa. Manusia akan lebih terlena oleh kemampuan gadget-nya sehingga lupa belajar untuk melakukan suatu aktivitas tanpa bantuan teknologi.

\section{ML on PHP}
Beberapa pendekatan yang dapat dimanfaatkan  untuk ML dalam PHP. Misalnya Algoritma, Validasi Silang, Jaringan Saraf Tiruan, Pra-pemrosesan, Ekstraksi Fitur, dan banyak lagi contoh lainnya.

\subsection{Fitur ML on PHP}
Beberapa fitur yang dapat diterapkan dengan ML on PHP:
\begin{enumerate}
\item Association rule Learning
\begin{itemize}
\item Apriori
Pembelajaran aturan asosiasi berdasarkan algoritma Apriori untuk sering melakukan penambangan item.
\begin{itemize}
	\item Constructor Parameters
	\item Train
	\item Predict
	\item Associating
	\item Frequent item sets
\end{itemize}
\end{itemize}
\item Classification
\begin{itemize}
\item SVC
\item k-Nearest Neighbors
\item Naive Bayes
\end{itemize}
\item Regression
\begin{itemize}
\item Least Squares
\item SVR
\end{itemize}
\item Clustering
\begin{itemize}
\item k-Means
\item DBSCAN
\end{itemize}
\item Metric
\begin{itemize}
\item Accuracy
\item Confusion Matrix
\item  Classification Report
\end{itemize}
\item Workflow
\begin{itemize}
\item  Pipeline
\end{itemize}
\item Neural Network
\begin{itemize}
\item Multilayer Perceptron Classifier
\end{itemize}
\item Cross Validation
\begin{itemize}
\item Random Split
\item Stratified Random Split
\end{itemize}
\item Feature Selection
\begin{itemize}
\item Variance Threshold
\item SelectKBest
\end{itemize}
\item Preprocessing
\begin{itemize}
\item Normalization
\item Imputation missing values
\end{itemize}
\item Feature Extraction
\begin{itemize}
\item Token Count Vectorizer
\item Tf-idf Transformer
\end{itemize}
\item Datasets
\begin{itemize}
\item Array
\item CSV
\item Files
\item SVM
\item MNIST
\end{itemize}
\item Models management
\begin{itemize}
\item Persistency
\end{itemize}
\item Math
\begin{itemize}
\item Distance
\item Matrix
\item Set
\item Statistic
\end{itemize}
\end{enumerate}


\subsection{Berkontribusi ke PHP-ML}
\par PHP-ML adalah proyek sumber terbuka. Jika Anda ingin berkontribusi, silakan baca teks berikut ini. Sebelum saya dapat menggabungkan Permintaan Tarik Anda, berikut adalah beberapa panduan yang perlu Anda ikuti. Panduan ini ada untuk tidak mengganggu Anda, tetapi untuk menjaga basis kode tetap bersih, terpadu, dan bukti di masa mendatang.
\begin{enumerate}
\item Cabang
Anda hanya harus membuka permintaan tarik terhadap cabang master.
\item Tes Unit
Coba tambahkan tes untuk permintaan tarik Anda. Anda dapat menjalankan Unit-test dengan skript:
	vendor/bin/phpunit
\item Tes Kinerja
Sebelum menjalankan skrip bootstrap pertama kali, akan mengunduh semua set data yang diperlukan dari repositori publik : php-ai/php-m-datasets.
Tes kinerja waktu:
vendor/bin/phpbench run--report=time
Tes kinerja memori:
vendor/bin/phpbench run --report=memory
\item Travis
GitHub secara otomatis menjalankan permintaan tarik Anda melalui Travis CI. Jika Anda melanggar tes, saya tidak dapat menggabungkan kode Anda, jadi pastikan kode Anda berfungsi sebelum membuka Permintaan Tarik.
\item Menggabungkan
Tolong beri saya waktu untuk meninjau permintaan tarik Anda. Saya akan memberikan yang terbaik untuk meninjau semuanya secepat mungkin, tetapi tidak selalu sesuai dengan harapan saya.
\item Standar Pengkodean \& Analisis Statis
Saat berkontribusi kode ke PHP-ML, Anda harus mengikuti standar pengkodeannya. Untuk melakukannya, jalankan:
	composer fix-cs
\end{enumerate}










\chapter{Studi Kasus dan Penyelesaian}
\input{chapters/5}

\chapter{Judul Bagian Keenam}
\input{chapters/6}



\bibliographystyle{IEEEtran} 
%\def\bibfont{\normalsize}
\bibliography{references}


%%%%%%%%%%%%%%%
%%  The default LaTeX Index
%%  Don't need to add any commands before \begin{document}
\printindex

%%%% Making an index
%% 
%% 1. Make index entries, don't leave any spaces so that they
%% will be sorted correctly.
%% 
%% \index{term}
%% \index{term!subterm}
%% \index{term!subterm!subsubterm}
%% 
%% 2. Run LaTeX several times to produce <filename>.idx
%% 
%% 3. On command line, type  makeindx <filename> which
%% will produce <filename>.ind 
%% 
%% 4. Type \printindex to make the index appear in your book.
%% 
%% 5. If you would like to edit <filename>.ind 
%% you may do so. See docs.pdf for more information.
%% 
%%%%%%%%%%%%%%%%%%%%%%%%%%%%%%

%%%%%%%%%%%%%% Making Multiple Indices %%%%%%%%%%%%%%%%
%% 1. 
%% \usepackage{multind}
%% \makeindex{book}
%% \makeindex{authors}
%% \begin{document}
%% 
%% 2.
%% % add index terms to your book, ie,
%% \index{book}{A term to go to the topic index}
%% \index{authors}{Put this author in the author index}
%% 
%% \index{book}{Cows}
%% \index{book}{Cows!Jersey}
%% \index{book}{Cows!Jersey!Brown}
%% 
%% \index{author}{Douglas Adams}
%% \index{author}{Boethius}
%% \index{author}{Mark Twain}
%% 
%% 3. On command line type 
%% makeindex topic 
%% makeindex authors
%% 
%% 4.
%% this is a Wiley command to make the indices print:
%% \multiprintindex{book}{Topic index}
%% \multiprintindex{authors}{Author index}

\end{document}



